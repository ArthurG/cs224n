\documentclass[11pt,largemargins]{homework}

\newcommand{\hwname}{}
\newcommand{\hwemail}{arthurg}
\newcommand{\hwtype}{}
\newcommand{\hwnum}{}
\newcommand{\hwclass}{cs224n}
\newcommand{\hwlecture}{a5}
\newcommand{\hwsection}{}

% This is just used to generate filler content. You don't need it in an actual
% homework!
\usepackage{lipsum}

\begin{document}
\maketitle

\question

When using convolutions on character level language models, the convolutions (of size k) are able to operate on words of arbitrary length. However, the output of this convolution will be a vector of size $len - k$. 

\question

The minimum $w_{word}$ is 1. After padding the $sow$ and $eow$ tokens,  the minimum length for $x_{padded}$ would thus be $R^{1+2} = R^{3}$

To ensure we apply at least one full convolution, we need to padd $x_{padded}$ to size 5. Then, we need padding of 1. $x_{reshaped} \in R^{e_{char} x 1 + 2 + (2 * 1)} =  R^{e_{char} x 5}$

\question
It's useful for the extremes of $x_{gate}$ to set $x_{highway}$ be either fully $x_{proj}$ or fully $x_{conv\_out}$ because it allows certain character embeddings to optionally pass through another layer. 

It's probably a better idea to set the bias to positive. This will ensure $x_{gate} -> 1$. If $x_{gate} = 0$, we will have no gradient on $x_{proj}$ which makes the layer useless. 

\question

- Parallizes better on GPUs

- Multi-headed attention might improve translation accuracy when trying to do things like verb - noun agreements 

\question 

I tested my Highway network by:

1. Running a batch of $x_{conv\_out}$ to check the dimensions are correct 

2. Set the weights of $w$, and send in an $x_{conv\_out}$ vector (of size 5). Manually do the matrix math to check the output matches. In these test cases, I made sure to find cases where the sigmoid is 0, sigmoid is 1, and where the relu is 0. 

I'm confident that these two tests will cover the edge cases. In general, since I used pre-defined components in PyTorch, most of the edge cases are handled for me. 
 
\end{document}
